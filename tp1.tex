\documentclass[10pt,a4paper]{article}

\usepackage[spanish,activeacute,es-tabla]{babel}
\usepackage[utf8]{inputenc}
\usepackage{ifthen}
\usepackage{listings}
\usepackage{dsfont}
\usepackage{subcaption}
\usepackage{amsmath}
\usepackage[strict]{changepage}
\usepackage[top=1cm,bottom=2cm,left=1cm,right=1cm]{geometry}%
\usepackage{color}%
\newcommand{\tocarEspacios}{%
	\addtolength{\leftskip}{3em}%
	\setlength{\parindent}{0em}%
}

% Especificacion de procs

\newcommand{\In}{\textsf{in }}
\newcommand{\Out}{\textsf{out }}
\newcommand{\Inout}{\textsf{inout }}

\newcommand{\encabezadoDeProc}[4]{%
	% Ponemos la palabrita problema en tt
	%  \noindent%
	{\normalfont\bfseries\ttfamily proc}%
	% Ponemos el nombre del problema
	\ %
	{\normalfont\ttfamily #2}%
	\
	% Ponemos los parametros
	(#3)%
	\ifthenelse{\equal{#4}{}}{}{%
		% Por ultimo, va el tipo del resultado
		\ : #4}
}

\newenvironment{proc}[4][res]{%
	
	% El parametro 1 (opcional) es el nombre del resultado
	% El parametro 2 es el nombre del problema
	% El parametro 3 son los parametros
	% El parametro 4 es el tipo del resultado
	% Preambulo del ambiente problema
	% Tenemos que definir los comandos requiere, asegura, modifica y aux
	\newcommand{\requiere}[2][]{%
		{\normalfont\bfseries\ttfamily requiere}%
		\ifthenelse{\equal{##1}{}}{}{\ {\normalfont\ttfamily ##1} :}\ %
		\{\ensuremath{##2}\}%
		{\normalfont\bfseries\,\par}%
	}
	\newcommand{\asegura}[2][]{%
		{\normalfont\bfseries\ttfamily asegura}%
		\ifthenelse{\equal{##1}{}}{}{\ {\normalfont\ttfamily ##1} :}\
		\{\ensuremath{##2}\}%
		{\normalfont\bfseries\,\par}%
	}
	\renewcommand{\aux}[4]{%
		{\normalfont\bfseries\ttfamily aux\ }%
		{\normalfont\ttfamily ##1}%
		\ifthenelse{\equal{##2}{}}{}{\ (##2)}\ : ##3\, = \ensuremath{##4}%
		{\normalfont\bfseries\,;\par}%
	}
	\renewcommand{\pred}[3]{%
		{\normalfont\bfseries\ttfamily pred }%
		{\normalfont\ttfamily ##1}%
		\ifthenelse{\equal{##2}{}}{}{\ (##2) }%
		\{%
		\begin{adjustwidth}{+5em}{}
			\ensuremath{##3}
		\end{adjustwidth}
		\}%
		{\normalfont\bfseries\,\par}%
	}
	
	\newcommand{\res}{#1}
	\vspace{1ex}
	\noindent
	\encabezadoDeProc{#1}{#2}{#3}{#4}
	% Abrimos la llave
	\par%
	\tocarEspacios
}
{
	% Cerramos la llave
	\vspace{1ex}
}

\newcommand{\aux}[4]{%
	{\normalfont\bfseries\ttfamily\noindent aux\ }%
	{\normalfont\ttfamily #1}%
	\ifthenelse{\equal{#2}{}}{}{\ (#2)}\ : #3\, = \ensuremath{#4}%
	{\normalfont\bfseries\,;\par}%
}

\newcommand{\pred}[3]{%
	{\normalfont\bfseries\ttfamily\noindent pred }%
	{\normalfont\ttfamily #1}%
	\ifthenelse{\equal{#2}{}}{}{\ (#2) }%
	\{%
	\begin{adjustwidth}{+2em}{}
		\ensuremath{#3}
	\end{adjustwidth}
	\}%
	{\normalfont\bfseries\,\par}%
}

% Tipos

\newcommand{\nat}{\ensuremath{\mathds{N}}}
\newcommand{\ent}{\ensuremath{\mathds{Z}}}
\newcommand{\real}{\ensuremath{\mathds{R}}}
\newcommand{\float}{\ensuremath{\mathds{R}}}
\newcommand{\bool}{\ensuremath{\mathsf{Bool}}}
\newcommand{\cha}{\ensuremath{\mathsf{Char}}}
\newcommand{\str}{\ensuremath{\mathsf{String}}}
\newcommand{\metavariable}{\ensuremath{\mathsf{v_{0}}}}
\newcommand{\precondicion}{\ensuremath{\mathsf{P_{c}}}}
\newcommand{\funcionDecresiente}{\ensuremath{\mathsf{f_{v}}}}
\newcommand{\postcondicion}{\ensuremath{\mathsf{Q_{c}}}}
\newcommand{\reemplazo}{\ensuremath{\mathsf{I_{i+1}^i}}}

\newcommand{\apuestac}{\ensuremath{\mathsf{apuesta_{c}}}}
\newcommand{\apuestas}{\ensuremath{\mathsf{apuesta_{s}}}}
\newcommand{\pagoc}{\ensuremath{\mathsf{pago_{c}}}}
\newcommand{\pagos}{\textit{\ensuremath{\mathsf{pago_{s}}}}}

% Logica

\newcommand{\True}{\ensuremath{\mathrm{true}}}
\newcommand{\False}{\ensuremath{\mathrm{false}}}
\newcommand{\Then}{\ensuremath{\rightarrow}}
\newcommand{\Iff}{\ensuremath{\leftrightarrow}}
\newcommand{\implica}{\ensuremath{\longrightarrow}}
\newcommand{\IfThenElse}[3]{\ensuremath{\mathsf{if}\ #1\ \mathsf{then}\ #2\ \mathsf{else}\ #3\ \mathsf{fi}}}
\newcommand{\y}{\land}
\newcommand{\yLuego}{\land _L}
\newcommand{\oLuego}{\lor _L}
\newcommand{\implicaLuego}{\implica _L}

\newcommand{\cuantificador}[5]{%
	\ensuremath{(#2 #3: #4)\ (%
		\ifthenelse{\equal{#1}{unalinea}}{
			#5
		}{
			$ % exiting math mode
			\begin{adjustwidth}{+2em}{}
				$#5$%
			\end{adjustwidth}%
			$ % entering math mode
		}
		)}
}

\newcommand{\existe}[4][]{%
	\cuantificador{#1}{\exists}{#2}{#3}{#4}
}
\newcommand{\paraTodo}[4][]{%
	\cuantificador{#1}{\forall}{#2}{#3}{#4}
}

%listas

\newcommand{\TLista}[1]{\ensuremath{seq \langle #1\rangle}}
\newcommand{\lvacia}{\ensuremath{[\ ]}}
\newcommand{\lv}{\ensuremath{[\ ]}}
\newcommand{\longitud}[1]{\ensuremath{|#1|}}
\newcommand{\cons}[1]{\ensuremath{\mathsf{addFirst}}(#1)}
\newcommand{\indice}[1]{\ensuremath{\mathsf{indice}}(#1)}
\newcommand{\conc}[1]{\ensuremath{\mathsf{concat}}(#1)}
\newcommand{\cab}[1]{\ensuremath{\mathsf{head}}(#1)}
\newcommand{\cola}[1]{\ensuremath{\mathsf{tail}}(#1)}
\newcommand{\sub}[1]{\ensuremath{\mathsf{subseq}}(#1)}
\newcommand{\en}[1]{\ensuremath{\mathsf{en}}(#1)}
\newcommand{\cuenta}[2]{\mathsf{cuenta}\ensuremath{(#1, #2)}}
\newcommand{\suma}[1]{\mathsf{suma}(#1)}
\newcommand{\twodots}{\ensuremath{\mathrm{..}}}
\newcommand{\masmas}{\ensuremath{++}}
\newcommand{\matriz}[1]{\TLista{\TLista{#1}}}
\newcommand{\seqchar}{\TLista{\cha}}

\renewcommand{\lstlistingname}{Código}
\lstset{% general command to set parameter(s)
	language=Java,
	morekeywords={endif, endwhile, skip},
	basewidth={0.47em,0.40em},
	columns=fixed, fontadjust, resetmargins, xrightmargin=5pt, xleftmargin=15pt,
	flexiblecolumns=false, tabsize=4, breaklines, breakatwhitespace=false, extendedchars=true,
	numbers=left, numberstyle=\tiny, stepnumber=1, numbersep=9pt,
	frame=l, framesep=3pt,
	captionpos=b,
}

\usepackage{caratula}
\usepackage[spanish]{babel} 
\usepackage{amsfonts} 
\usepackage[dvipsnames]{xcolor}
\usepackage{graphicx}
\usepackage{enumitem}

\titulo{Trabajo práctico - Programación Lineal}
%\subtitulo{Subtítulo del tp}

\fecha{\today}

\materia{Intro a IO y Optimización}


\integrante{Lamonica, Ivo}{66/22}{Completaer}
\integrante{Masetto, Lautaro}{1052/22}{Completaer}
\integrante{Vanotti, Franco}{464/23}{fvanotti15@gmail.com}

\graphicspath{{../static/}}

\setcounter{section}{-1}

\begin{document}

\maketitle



\section{Introducción} 

Una refinería produce tres productos, combustible para aviones, combustible para vehículos y kerosene. Gracias a
los buenos precios de venta que mantiene, la refinería vende todos los productos que produce. \\ 

El proceso de producción se compone de refinado, fraccionado y embalaje. Para realizar estas tareas, la empresa cuenta con 4 sectores:


\begin{itemize}
	\item refinado: capacidad mensual de 38.000 horas, gasto fijo de \$5.000.000.
	\item fraccionado: capacidad mensual de 80.000 horas, gasto fijo de \$5.000.000.
	\item embalaje de combustible para aviones: capacidad mensual de 4.000 horas, gasto fijo de \$2.000.000.
	\item embalaje de combustible para vehículos: capacidad mensual de 6.000 horas, gasto fijo de \$1.000.000.
	\item embalaje de kerosene: capacidad mensual de 7.000 horas, gasto fijo de \$500.000.
\end{itemize}

El tiempo requerido para refinar 1000 litros de combustible para aviones es de 10 horas, mientras que para fraccionarlos son necesarias 20 horas y 4 para su embalaje. 
Para el combustible para vehículos los tiempos son de 5, 10 y 2 horas respectivamente, y para los 1000 litros de kerosene de 3, 6 y 1 hora. \\ 

El precio de venta es de \$16000 para los mil litros de combustible para aviones, de \$8000 para los mil litros de
combustible para vehículos y de \$4000 los mil litros de kerosene. El costo de la materia prima para mil litros de
combustible para aviones es de \$4000, el de refinado de \$4100, el de fraccionado de \$1000, y el de embalaje de
\$1000. Para 1000 litros de combustible para vehículos, los costos son de \$1000, \$3000, \$600 y \$500, respectivamente.
Y para 1000 litros de kerosene de \$500, \$1500, \$400 y \$400. \\

En el último mes, la refinería produjo 500.000 litros de combustible para aviones, 3.000.000 litros de combustible
para vehículos y 6.000.000 litros de kerosene, lo que implicí una pérdida en el combustible para aviones al considerar
prorrateados los gastos fijos. \\

El gerente de ventas señaló que estudios de mercado indican que no es posible aumentar el precio de venta del combustible para aviones y, por lo tanto,
para aumentar las ganancias de la empresa se debe discontinuar su producción. \\

Sin embargo, la conclusión del jefe del departamento de embalaje de combustible para aviones es distinta. Según él,
los gastos fijos de su sector inciden tanto en el costo de cada litro de combustible para aviones porque se produce
poco de este producto, y para aumentar las ganancias de la empresa propone que se produzca más. \\

En vista de las distintas propuestas, el director de la empresa necesita ayuda para evaluarlas y tomar las decisiones
apropiadas. Para esto, analizar los siguientes ítems, \textbf{evitando reoptimizar en todos los casos que sea posible:}

\clearpage

\section{Ejercicios} 

\begin{enumerate}
    \item Calcular la ganancia o pérdida (prorrateando los gastos fijos) de cada producto que se obtuvo en el mes
    anterior (cuando se produjeron 500.000 litros de combustible para aviones, 3.000.000 de combustible para
    vehículos y 6.000.000 litros de kerosene) y la ganancia (o pérdida) total de la compañía. \\

    \textbf{Calcular Costos Variables:} Usamos los costos por cada 1.000 litros y multiplicamos por la producción total para cada producto. \\

    - \underline{Combustible para aviones} (500.000 L / 1.000 = 500 unidades):

    \begin{itemize}

        \item Materia prima: 500 × 4.000 = \$2.000.000
        \item Refinado: 500 × 4.100 = \$2.050.000
        \item Fraccionado: 500 × 1.000 = \$500.000
        \item Embalaje: 500 × 1.000 = \$500.000 \\
        \item Total: \$5.050.000

    \end{itemize}

    - \underline{Combustible para vehículos} (3.000.000 L / 1.000 = 3.000 unidades):

    \begin{itemize}

        \item Materia prima: 3.000 × 1.000 = \$3.000.000
        \item Refinado: 3.000 × 3.000 = \$9.000.000
        \item Fraccionado: 3.000 × 600 = \$1.800.000
        \item Embalaje: 3.000 × 500 = \$1.500.000 \\
        \item Total: \$15.300.000

    \end{itemize}

    - \underline{Kerosene} (6.000.000 L / 1.000 = 6.000 unidades):

    \begin{itemize}

        \item Materia prima: 6.000 × 500 = \$3.000.000
        \item Refinado: 6.000 × 1.500 = \$9.000.000
        \item Fraccionado: 6.000 × 400 = \$2.400.000
        \item Embalaje: 6.000 × 400 = \$2.400.000 \\
        \item Total: \$16.800.000\\

    \end{itemize}

    \textbf{Calcular Ingresos por Ventas:}

    \begin{itemize}

        \item Aviones: 500 × 16.000 = \$8.000.000
        \item Vehículos: 3.000 × 8.000 = \$24.000.000
        \item Kerosene: 6.000 × 4.000 = \$24.000.000 \\

    \end{itemize}

    \textbf{Ganancia Bruta por Producto:}

    \begin{center}
    \scalebox{1}{
    \begin{tabular}{|c|c|c|c|}
        \hline
        \textbf{Producto} & \textbf{Ingreso} & \textbf{Costo} & \textbf{Ganancia bruta} \\
        \hline
        Combustible Aviones & 8.000.000 & 5.050.000 & 2.950.000 \\
        \hline
        Combustible Vehículos & 24.000.000 & 15.300.000 & 8.700.000 \\
        \hline
        Kerosene & 24.000.000 & 16.800.000 & 7.200.000 \\
        \hline
    \end{tabular}
    }
    \end{center}
    
    \clearpage

    \textbf{Prorratear Gastos Fijos:} \\

    \underline{Gasto Fijo Total}:

    \begin{itemize}

        \item Refinado: \$5.000.000
        \item Fraccionado: \$5.000.000
        \item Embalaje aviones: \$2.000.000
        \item Embalaje vehículos: \$1.000.000
        \item Embalaje kerosene: \$500.000 \\
        \item Total: \$13.500.000 \\

    \end{itemize}

    \underline{Prorrateamos los gastos según uso de horas}: \\

    Primero, calculamos las horas usadas por cada producto en cada etapa (por cada 1000 litros)

    \begin{center}
        \scalebox{1}{
        \begin{tabular}{|c|c|c|c|c|}
            \hline
            \textbf{Producto} & \textbf{Unidades} & \textbf{Refinado (h)} & \textbf{Fraccionado (h)} & \textbf{Embalaje (h)}\\
            \hline
            Combustible Aviones & 500 & 10 × 500 = 5.000 & 20 × 500 = 10.000 & 4 × 500 = 2.000\\
            \hline
            Combustible Vehículos & 3.000 & 5 × 3.000 = 15.000 & 10 × 3.000 = 30.000 & 2 × 3.000 = 6.000\\
            \hline
            Kerosene & 6.000 & 3 × 6.000 = 18.000 & 6 × 6.000 = 36.000 & 	1 × 6.000 = 6.000\\
            \hline
        \end{tabular}
        }
    \end{center}

    \vspace{0.5em}

    \begin{enumerate}[label=\alph*)]

        \item Refinado (38.000 horas totales):
        
        \begin{itemize}

            \item Aviones: 5.000/38.000 × 5.000.000 = \$657.895
            \item Vehículos: 15.000/38.000 × 5.000.000 = \$1.973.684
            \item Kerosene: 18.000/38.000 × 5.000.000 = \$2.368.421 \\
    
        \end{itemize}

        \item Fraccionado (76.000 horas totales):
        
        \begin{itemize}

            \item Aviones: 10.000/76.000 × 5.000.000 = \$657.895
            \item Vehículos: 30.000/76.000 × 5.000.000 = \$1.973.684
            \item Kerosene: 36.000/76.000 × 5.000.000 = \$2.368.421 \\
    
        \end{itemize}

        \item Embalaje (por sector):
        
        \begin{itemize}

            \item Aviones: \$2.000.000
            \item Vehículos: \$1.000.000
            \item Kerosene: \$500.000\\
    
        \end{itemize}

    \end{enumerate}

    \vspace{0.5em}

    \underline{Gasto Fijo Total por Producto}: \\

    \begin{center}
        \scalebox{1}{
        \begin{tabular}{|c|c|c|c|c|}
            \hline
            \textbf{Producto} & \textbf{Refinado} & \textbf{Fraccionado} & \textbf{Embalaje} & \textbf{Gasto Fijo Total} \\
            \hline
            Combustible Aviones & 657.895 & 657.895 & 2.000.000 & 3.315.790\\
            \hline
            Combustible Vehículos & 1.973.684 & 1.973.684 & 1.000.000 & 4.947.368\\
            \hline
            Kerosene & 2.368.421 & 2.368.421 & 500.000 & 5.236.842\\
            \hline
        \end{tabular}
        }
    \end{center}

    \vspace{0.5em}

    \textbf{Ganancia Neta por Producto:} 

    \begin{center}
        \scalebox{1}{
        \begin{tabular}{|c|c|c|c|}
            \hline
            \textbf{Producto} & \textbf{Ganancia bruta} & \textbf{Gasto fijo} & \textbf{Ganancia neta} \\
            \hline
            Combustible Aviones & 2.950.000 & 3.315.790 & -365.790 \\
            \hline
            Combustible Vehículos & 8.700.000 & 4.947.368 & 3.752.632 \\
            \hline
            Kerosene & 7.200.000 & 5.236.842 & 	1.963.158 \\
            \hline
        \end{tabular}
        }
    \end{center}

    \vspace{0.5em}

    \textbf{Ganancia Total de la Empresa: } -\$365.790 + \$3.752.632 + \$1.963.158 = \$5.350.000

    \item Si la empresa no hubiese producido combustible para aviones manteniendo en los mismos valores los otros
    productos, ¿la ganancia de la compañía habría sido mejor? Suponer que se cierra el sector de embalaje de
    combustibles para aviones. 

    \vspace{0.5em}

    \textbf{Datos Relevantes:}

    \begin{itemize}

        \item Se elimina completamente el combustible para aviones.
        \item Se mantiene igual la producción de:
        
        \begin{itemize}

            \item Combustible para vehículos: 3.000.000 L
            \item Kerosene: 6.000.000 L
    
        \end{itemize}

        \item El sector de embalaje para aviones se cierra, así que se eliminan sus gastos fijos (\$2.000.000).
        \item Los demás sectores siguen funcionando con los productos restantes. \\

    \end{itemize}

    Entonces, la ganacia bruta del combustible para vehículos y del kerosene siguen siendo: \\

    \begin{center}
        \scalebox{1}{
        \begin{tabular}{|c|c|c|c|}
            \hline
            \textbf{Producto} & \textbf{Ingreso} & \textbf{Costo} & \textbf{Ganancia bruta} \\
            \hline
            Combustible Vehículos & 24.000.000 & 15.300.000 & 8.700.000 \\
            \hline
            Kerosene & 24.000.000 & 16.800.000 & 7.200.000 \\
            \hline
        \end{tabular}
        }
    \end{center}

    \vspace{0.5em}

    Lo que cambian son los gastos fijos: \\

    \textbf{Capacidad Liberada:} \\

    Al eliminar la producción de combustiblede aviones:

    \begin{itemize}

        \item Refinado: Se liberan 5.000 h
        \item Fraccionado: Se liberan 10.000 h
        \item Embalaje Combustible Aviones: se cierra

    \end{itemize}

    \vspace{0.5em}

    \textbf{Prorratear Gastos Fijos:} \\

    \underline{Gasto Fijo Total}:

    \begin{itemize}

        \item Refinado: \$5.000.000
        \item Fraccionado: \$5.000.000
        \item Embalaje vehículos: \$1.000.000
        \item Embalaje kerosene: \$500.000 \\
        \item Total: \$11.500.000 \\

    \end{itemize}

    \underline{Prorrateamos los gastos según uso de horas}: \\

    Primero, calculamos las horas usadas por cada producto en cada etapa (por cada 1000 litros)

    \begin{center}
        \scalebox{1}{
        \begin{tabular}{|c|c|c|c|c|}
            \hline
            \textbf{Producto} & \textbf{Unidades} & \textbf{Refinado (h)} & \textbf{Fraccionado (h)} & \textbf{Embalaje (h)}\\
            \hline
            Combustible Vehículos & 3.000 & 5 × 3.000 = 15.000 & 10 × 3.000 = 30.000 & 2 × 3.000 = 6.000\\
            \hline
            Kerosene & 6.000 & 3 × 6.000 = 18.000 & 6 × 6.000 = 36.000 & 	1 × 6.000 = 6.000\\
            \hline
        \end{tabular}
        }
    \end{center}

    \clearpage

    \begin{enumerate}[label=\alph*)]

        \item Refinado (33.000 horas totales):
        
        \begin{itemize}

            \item Vehículos: 15.000/33.000 × 5.000.000 = \$2.272.727
            \item Kerosene: 18.000/33.000 × 5.000.000 = \$2.727.273 \\
    
        \end{itemize}

        \item Fraccionado (66.000 horas totales):
        
        \begin{itemize}

            \item Vehículos: 30.000/66.000 × 5.000.000 = \$2.272.727
            \item Kerosene: 36.000/6.000 × 5.000.000 = \$2.727.273 \\
    
        \end{itemize}

        \item Embalaje (por sector):
        
        \begin{itemize}

            \item Vehículos: \$1.000.000
            \item Kerosene: \$500.000 \\
    
        \end{itemize}

    \end{enumerate}

    \vspace{0.5em}

    \underline{Gasto Fijo Total por Producto}: \\

    \begin{center}
        \scalebox{1}{
        \begin{tabular}{|c|c|c|c|c|}
            \hline
            \textbf{Producto} & \textbf{Refinado} & \textbf{Fraccionado} & \textbf{Embalaje} & \textbf{Gasto Fijo Total} \\
            \hline
            Combustible Vehículos & 2.272.727 & 2.272.727 & 1.000.000 & 5.545.454\\
            \hline
            Kerosene & 2.727.273 & 2.727.273 & 500.000 & 5.954.546\\
            \hline
        \end{tabular}
        }
    \end{center}

    \vspace{0.5em}

    \textbf{Ganancia Neta por Producto:} 

    \begin{center}
        \scalebox{1}{
        \begin{tabular}{|c|c|c|c|}
            \hline
            \textbf{Producto} & \textbf{Ganancia bruta} & \textbf{Gasto fijo} & \textbf{Ganancia neta} \\
            \hline
            Combustible Vehículos & 8.700.000 &  5.545.454 & 3.154.546 \\
            \hline
            Kerosene & 7.200.000 & 5.954.546 & 	1.245.454 \\
            \hline
        \end{tabular}
        }
    \end{center}

    \vspace{0.5em}

    \textbf{Ganancia Total de la Empresa: } \$3.154.546 + \$1.245.454  = \$4.400.000

    \vspace{0.5em}

    \textbf{Conclusión }:

    La ganancia habría sido menor si no se producía combustible para aviones. \\
    Incluso aunque el sector de embalaje para combustilbe para avion se cerrara y sus gastos se eliminaran, la empresa perdería aproximadamente \$2 millones de ganancia. 

    Esto a priori refuerza el argumento del jefe del área de embalaje, quien propone aumentar la producción de combustible para aviones, 
    ya que diluye el impacto de los costos fijos sobre cada litro y mejora la ganancia total. \\
    
    \item ¿Y si hubiese aumentado lo máximo posible la producción de los otros productos? Suponer que se cierra el
    sector de embalaje de combustibles para aviones.

    Esto implica resolver un problema de programación lineal para maximizar la \textbf{ganancia}, teniendo en cuenta:

    \begin{itemize}

        \item Restricciones de horas por sector
        \item Ganancias por producto.
        \item No se puede producir combustible para aviones. \\

    \end{itemize}

    \textbf{Variables}

    \begin{itemize}

        \item $x_{1}$: miles de litros de combustible para aviones (no se produce, así que se elimina)
        \item $x_{2}$: miles de litros de combustible para vehículos.
        \item $x_{3}$: miles de litros de kerosene. \\

    \end{itemize}

    \clearpage

    \textbf{Función objetivo: Maximizar la ganancia total}

    \vspace{1em}

    \underline{Ganancia por 1.000 litros}:

    \begin{itemize}

        \item Combustible para vehículos:
        \begin{itemize}

            \item Precio de Venta: \$8.000.
            \item Costos variables: \$1.000 + \$3.000 + \$600 + \$500 = \$5.100.
            \item Ganancia por 1.000 litros: \$2.900. \\
    
        \end{itemize}

        \item Kerosene:
        \begin{itemize}

            \item Precio de Venta: \$4.000.
            \item Costos variables: \$500 + \$1.500 + \$400 + \$400 = \$2.800.
            \item Ganancia por 1.000 litros: \$1.200. \\
    
        \end{itemize}

    \end{itemize}

    \underline{Función objetivo}: 

    \begin{center}
        
        Maximizar 2.900 $x_{2}$ + 1.200 $x_{3}$

    \end{center}

    \underline{Restricciones}: 

    \begin{itemize}

        \item Refinado (máximo 38.000 h):  5 $x_{2}$ + 3 $x_{3}$ $\leq$ 38.000
        \item Fraccionado (máximo 80.000 h): 10 $x_{2}$ + 6 $x_{3}$ $\leq$ 80.000
        \item Embalaje Combustible Vehículos (máximo 6.000 h): 2 $x_{2}$ $\leq$ 6.000 $\rightarrow$ $x_{2}$ $\leq$ 3.000.
        \item Embalaje K (máximo 7.000 h): $x_{3}$ $\leq$ 7.000
        \item Condiciones de no negatividad: $x_{2}$  $\geq$ 0, $x_{3}$ $\geq$ 0\\

    \end{itemize}

    \underline{Solucion del Modelo}: 

    \begin{itemize}

        \item $x_{2}$ = 3.000
        \item $x_{3}$ = 7.000
        \item $Z$ = \$17.100.000\\

    \end{itemize}

    \underline{Verificamos si cumple las restricciones}:

    \begin{itemize}

        \item Refinado (máximo 38.000 h):  5 x 3.000 + 3 x 7.000 = 36.000 $\leq$ 38.000
        \item Fraccionado (máximo 80.000 h): 10 x 3.000 + 6 x 7.000 = 72.000 $\leq$ 80.000
        \item Embalaje Combustible Vehículos (máximo 6.000 h): 3.000 $\leq$ 3.000.
        \item Embalaje K (máximo 7.000 h): 7.000 $\leq$ 7.000
        \item Condiciones de no negatividad: 3.000  $\geq$ 0, 7.000 $\geq$ 0\\

    \end{itemize}

    Ahora que verificamos que es una solucion factible, tenemos que restarle los gastos fijos a z para determinar la ganancia total de la compañía.\\

    \textbf{Gastos Fijos}:

    \begin{itemize}

        \item Refinado: \$5.000.000.
        \item Fraccionado: \$5.000.000
        \item Embalaje Combustible Vehículos: \$1.000.000.
        \item Embalaje Kerosene: \$500.000.
        \item Total: \$11.500.000.

    \end{itemize}

    Luego, la ganancia total de la compañía es: \$17.100.000 - \$11.500.000 = \$5.600.000. \\

    \textbf{Conclusión }:

    La ganancia habría sido mayor si no se producía combustible para aviones pero se aumetaba al máximo posible 
    la producción de los otros productos. 

    Esto a priori refuerza el argumento del gerente de ventas, quien propone discontinuar la producción de combustible para aviones, 
    ya que no es posible aumentar el precio de venta del mismo y de esta forma evitar perdidas.

    Sin embargo, no se sabe si la cantidad producida el mes pasado de cada producto fue la óptima, por lo que quizas no es que no convenga no producir
    combustible para avion sino que quizas conviene producir distintas cantidades de los porductos para poder generar mas ganancias.


    \item Determinar la cantidad óptima de producción mensual de cada producto para maximizar la ganancia de la compañía.
    
    Para determinar la cantidad óptima de producción se pueden plantea un modelo de programación lineal asumiendo que todos los sectores producen algo.\\

    \textbf{Variables}

    \begin{itemize}

        \item $x_{1}$: miles de litros de combustible para aviones.
        \item $x_{2}$: miles de litros de combustible para vehículos.
        \item $x_{3}$: miles de litros de kerosene. \\

    \end{itemize}

    \textbf{Función objetivo: Maximizar la ganancia total}

    \vspace{0.5em}

    \underline{Ganancia por 1.000 litros}:

    \begin{itemize}

        \item Combustible para aviones:
        \begin{itemize}

            \item Precio de Venta: \$16.000.
            \item Costos variables: \$4.000 + \$4.100 + \$1.000 + \$1.000 = \$10.100.
            \item Ganancia por 1000 litros: \$5.900. \\
    
        \end{itemize}

        \item Combustible para vehículos:
        \begin{itemize}

            \item Precio de Venta: \$8.000.
            \item Costos variables: \$1.000 + \$3.000 + \$600 + \$500 = \$5.100.
            \item Ganancia por 1.000 litros: \$2.900. \\
    
        \end{itemize}

        \item Kerosene:
        \begin{itemize}

            \item Precio de Venta: \$4.000.
            \item Costos variables: \$500 + \$1.500 + \$400 + \$400 = \$2.800.
            \item Ganancia por 1.000 litros: \$1.200. \\
    
        \end{itemize}

    \end{itemize}


    \underline{Función objetivo}: 

    \begin{center}
        
        Maximizar 5.900 $x_{1}$ + 2.900 $x_{2}$ + 1.200 $x_{3}$

    \end{center}


    \underline{Restricciones}: 

    \begin{itemize}

        \item Refinado (máximo 38.000 h): 10 $x_{1}$ + 5 $x_{2}$ + 3 $x_{3}$ $\leq$ 38.000
        \item Fraccionado (máximo 80.000 h): 20 $x_{1}$ + 10 $x_{2}$ + 6 $x_{3}$ $\leq$ 80.000
        \item Embalaje Combustible Aviones (máximo 4.000 h): 4 $x_{1}$ $\leq$ 4.000 $\rightarrow$ $x_{1}$ $\leq$ 1.000.
        \item Embalaje Combustible Vehículos (máximo 6.000 h): 2 $x_{2}$ $\leq$ 6.000 $\rightarrow$ $x_{2}$ $\leq$ 3.000.
        \item Embalaje K (máximo 7.000 h): $x_{3}$ $\leq$ 7.000
        \item Condiciones de no negatividad: $x_{1}$ $\geq$ 0, $x_{2}$ $\geq$ 0, $x_{3}$ $\geq$ 0\\

    \end{itemize}

    \clearpage

    \underline{Solucion del Modelo}: 

    \begin{itemize}

        \item $x_{1}$ = 1.000
        \item $x_{2}$ = 3.000
        \item $x_{3}$ = 4.333,333333333333
        \item $Z$ = \$19.800.000\\

    \end{itemize}

    \underline{Verificamos si cumple las restricciones}:

    \begin{itemize}

        \item Refinado (máximo 38.000 h): 10 x 1.000 + 5 x 3.000 + 3 x 4.333,333333333333 = 38.000 $\leq$ 38.000
        \item Fraccionado (máximo 80.000 h): 20 x 1.000 + 10 x 3.000 + 6 x 4.333,333333333333 = 76.000 $\leq$ 80.000
        \item Embalaje Combustible Aviones (máximo 4.000 h): 1.000 $\leq$ 1.000.
        \item Embalaje Combustible Vehículos (máximo 6.000 h): 3.000 $\leq$ 3.000.
        \item Embalaje K (máximo 7.000 h): 4.333,333333333333 $\leq$ 7.000
        \item Condiciones de no negatividad: 1.000 $\geq$ 0, 3.000 $\geq$ 0, 4.333,333333333333 $\geq$ 0 \\

    \end{itemize}

    Calculamos los gastos fijos:\\

    \textbf{Gastos Fijos}:

    \begin{itemize}

        \item Refinado: \$5.000.000.
        \item Fraccionado: \$5.000.000
        \item Embalaje Combustible Aviones: \$2.000.000.
        \item Embalaje Combustible Vehículos: \$1.000.000.
        \item Embalaje K: \$500.000.
        \item Total: \$13.500.000. \\

    \end{itemize}

    Luego, la ganancia total de la compañía es: \$19.800.000 - \$13.500.000 = \$6.300.000. \\

    \textbf{Conclusión }:

    Como se puede observar, la cantidad óptima de producción mensual de combustible para aviones es mayor a cero, es 
    decir, la conclusión del jefe del área de embalaje, parece ser la acertada. \\
    Esto debido a que al aumentar la producción de combustible para aviones, se diluye el impacto de los costos fijos sobre cada litro y 
    mejora la ganancia total. \\


    \item Indicar al director estas cantidades, el costo por 1000 litros de cada producto (prorrateando los costos fijos)
     y la ganancia total de la empresa. \\

     La producción óptima queda de la siguiente manera:

     \begin{itemize}

        \item $x_{1}$ = 1.000
        \item $x_{2}$ = 3.000
        \item $x_{3}$ = 4.333,333333333333

    \end{itemize}


     \textbf{Calcular Costos Variables:} \\

    - \underline{Combustible para aviones} (1.000 unidades):

    \begin{itemize}

        \item Materia prima: 1.000 × 4.000 = \$4.000.000
        \item Refinado: 1.000 × 4.100 = \$4.100.000
        \item Fraccionado: 1.000 × 1.000 = \$1.000.000
        \item Embalaje: 1.000 × 1.000 = \$1.000.000 \\
        \item Total: \$10.100.000

    \end{itemize}

    - \underline{Combustible para vehículos} (3.000 unidades):

    \begin{itemize}

        \item Materia prima: 3.000 × 1.000 = \$3.000.000
        \item Refinado: 3.000 × 3.000 = \$9.000.000
        \item Fraccionado: 3.000 × 600 = \$1.800.000
        \item Embalaje: 3.000 × 500 = \$1.500.000 \\
        \item Total: \$15.300.000

    \end{itemize}

    - \underline{Kerosene} (4.333,333333333333 unidades):

    \begin{itemize}

        \item Materia prima: 4.333,333333333333 × 500 = \$2.166.667
        \item Refinado: 4.333,333333333333 × 1.500 = \$6.500.000
        \item Fraccionado: 4.333,333333333333 × 400 = \$1.733.333
        \item Embalaje: 4.333,333333333333 × 400 = \$1.733.333 \\
        \item Total: \$12.133.333\\

    \end{itemize}

    \textbf{Calcular Ingresos por Ventas:}

    \begin{itemize}

        \item Aviones: 1.000 × 16.000 = \$16.000.000
        \item Vehículos: 3.000 × 8.000 = \$24.000.000
        \item Kerosene: 4.333,333333333333 × 4.000 = \$17.333.333 \\

    \end{itemize}

    \vspace{0.5em}

    \textbf{Ganancia Bruta por Producto:}

    \begin{center}
    \scalebox{1}{
    \begin{tabular}{|c|c|c|c|}
        \hline
        \textbf{Producto} & \textbf{Ingreso} & \textbf{Costo} & \textbf{Ganancia bruta} \\
        \hline
        Combustible Aviones & 16.000.000 & 10.100.000 & 5.900.000 \\
        \hline
        Combustible Vehículos & 24.000.000 & 15.300.000 & 8.700.000 \\
        \hline
        Kerosene & 17.333.333 & 12.133.333 & 5.200.000 \\
        \hline
    \end{tabular}
    }
    \end{center}
    
    \vspace{1em}

    \textbf{Prorratear Gastos Fijos:} \\

    \underline{Gasto Fijo Total}:

    \begin{itemize}

        \item Refinado: \$5.000.000
        \item Fraccionado: \$5.000.000
        \item Embalaje aviones: \$2.000.000
        \item Embalaje vehículos: \$1.000.000
        \item Embalaje kerosene: \$500.000 \\
        \item Total: \$13.500.000 \\

    \end{itemize}

    \underline{Prorrateamos los gastos según uso de horas}: \\

    Primero, calculamos las horas usadas por cada producto en cada etapa (por cada 1000 litros)

    \begin{center}
        \scalebox{1}{
        \begin{tabular}{|c|c|c|c|c|}
            \hline
            \textbf{Producto} & \textbf{Unidades} & \textbf{Refinado (h)} & \textbf{Fraccionado (h)} & \textbf{Embalaje (h)}\\
            \hline
            Combustible Aviones & 1.000 & 10 × 1.000 = 10.000 & 20 × 1.000 = 20.000 & 4 × 1.000 = 4.000\\
            \hline
            Combustible Vehículos & 3.000 & 5 × 3.000 = 15.000 & 10 × 3.000 = 30.000 & 2 ×  = 6.000\\
            \hline
            Kerosene & 4.333,33 & 3 × 4.333,33 = 13.000 & 6 × 4.333,33 = 26.000 & 1 × 4.333,33 = 4.333,33\\
            \hline
        \end{tabular}
        }
    \end{center}

    \clearpage

    \begin{enumerate}[label=\alph*)]

        \item Refinado (38.000 horas totales):
        
        \begin{itemize}

            \item Aviones: 10.000/38.000 × 5.000.000 = \$1.315.790
            \item Vehículos: 15.000/38.000 × 5.000.000 = \$1.973.684
            \item Kerosene: 13.000/38.000 × 5.000.000 = \$1.710.526 \\
    
        \end{itemize}

        \item Fraccionado (76.000 horas totales):
        
        \begin{itemize}

            \item Aviones: 20.000/76.000 × 5.000.000 = \$1.315.790
            \item Vehículos: 30.000/76.000 × 5.000.000 = \$1.973.684
            \item Kerosene: 26.000/76.000 × 5.000.000 = \$1.710.526 \\
    
        \end{itemize}

        \item Embalaje (por sector):
        
        \begin{itemize}

            \item Aviones: \$2.000.000
            \item Vehículos: \$1.000.000
            \item Kerosene: \$500.000\\
    
        \end{itemize}

    \end{enumerate}

    \vspace{0.5em}

    \underline{Gasto Fijo Total por Producto}: \\

    \begin{center}
        \scalebox{1}{
        \begin{tabular}{|c|c|c|c|c|}
            \hline
            \textbf{Producto} & \textbf{Refinado} & \textbf{Fraccionado} & \textbf{Embalaje} & \textbf{Gasto Fijo Total} \\
            \hline
            Combustible Aviones & 1.315.790 & 1.315.790 & 2.000.000 & 4.631.580\\
            \hline
            Combustible Vehículos & 1.973.684 & 1.973.684 & 1.000.000 & 4.947.368\\
            \hline
            Kerosene & 1.710.526 & 1.710.526 & 500.000 & 3.921.052\\
            \hline
        \end{tabular}
        }
    \end{center}

    \vspace{0.5em}

    \textbf{Ganancia Neta por Producto:} 

    \begin{center}
        \scalebox{1}{
        \begin{tabular}{|c|c|c|c|}
            \hline
            \textbf{Producto} & \textbf{Ganancia bruta} & \textbf{Gasto fijo} & \textbf{Ganancia neta} \\
            \hline
            Combustible Aviones & 5.900.000 & 4.631.580 & 1.268.420 \\
            \hline
            Combustible Vehículos & 8.700.000 & 4.947.368 & 3.752.632 \\
            \hline
            Kerosene & 5.200.000 & 3.921.052 & 1.278.948 \\
            \hline
        \end{tabular}
        }
    \end{center}

    \vspace{0.5em}

    \textbf{Ganancia Total de la Empresa: } \$1.268.420 + \$3.752.632 + \$1.278.948 = \$6.300.000 \\

    \item Al escuchar esto, el gerente de producción propuso aumentar la producción contratando 500 horas extras al
    mes del personal del sector de fraccionado. Asesorar al director sobre esta propuesta.\\

    Como se puede observar en el inciso anterior, claramente no es rentable esta propuesta. Esto debido a que para producir
    la cantidad óptima para maximizar las ganancias de la compañía, no se utilizan las 80.000 horas de capacidad mensual de fraccionado pero sí se 
    utilizan las 38.000 horas de capacidad mensual de refinado. \\

    Por lo tanto, aunque se agreguen 500 horas extras al sector de fraccionado, no se va a poder producir mas debido a que ya se llego a la capacidad
    máxima de horas del personal del refinado.\\

    \item Otra propuesta del gerente es contratar 1000 horas extras al mes del personal del sector de refinado. Indicar
    al director si es conveniente aceptar esta nueva propuesta y hasta cuánto debería pagar por cada hora extra
    de este sector

    Para poder asesorar al director sobre esta propuesta, haciendo un análisis de sensibilidad, primero debemos plantear el problema 
    dual del modelo anteriorr.\\

    Si nosotros tenemos el modelo:\\

    \underline{Función objetivo}: 

    \begin{center}
        
        Maximizar 5.900 $x_{1}$ + 2.900 $x_{2}$ + 1.200 $x_{3}$

    \end{center}

    \clearpage

    \underline{Restricciones}: 

    \begin{itemize}

        \item Refinado (máximo 38.000 h): 10 $x_{1}$ + 5 $x_{2}$ + 3 $x_{3}$ $\leq$ 38.000
        \item Fraccionado (máximo 80.000 h): 20 $x_{1}$ + 10 $x_{2}$ + 6 $x_{3}$ $\leq$ 80.000
        \item Embalaje Combustible Aviones (máximo 4.000 h): 4 $x_{1}$ $\leq$ 4.000 $\rightarrow$ $x_{1}$ $\leq$ 1.000.
        \item Embalaje Combustible Vehículos (máximo 6.000 h): 2 $x_{2}$ $\leq$ 6.000 $\rightarrow$ $x_{2}$ $\leq$ 3.000.
        \item Embalaje K (máximo 7.000 h): $x_{3}$ $\leq$ 7.000
        \item Condiciones de no negatividad: $x_{1}$ $\geq$ 0, $x_{2}$ $\geq$ 0, $x_{3}$ $\geq$ 0\\

    \end{itemize}

    \underline{Solucion del Modelo}: 

    \begin{itemize}

        \item $x_{1}$ = 1.000
        \item $x_{2}$ = 3.000
        \item $x_{3}$ = 4.333,333333333333
        \item $Z$ = \$19.800.000\\

    \end{itemize}

    \vspace{0.5em}

    Entonces el \textbf{Dual} queda de la siguiente manera: \\

    \underline{Función objetivo}: 

    \begin{center}
        
        Maximizar 38.000 $y_{1}$ + 80.000 $y_{2}$ + 1.000 $y_{3}$ + 3.000 $y_{4}$ + 7.000 $y_{5}$

    \end{center}

    \underline{Restricciones}: 

    \begin{itemize}

        \item 10 $y_{1}$ + 20 $y_{2}$ + $y_{3}$ $\geq$ 5.900
        \item 5 $y_{1}$ + 10 $y_{2}$ + $y_{4}$ $\geq$ 2.900
        \item 3 $y_{1}$ + 6 $y_{2}$ + $y_{5}$ $\geq$ 1.200
        \item Condiciones de no negatividad: $y_{1}$ $\geq$ 0, $y_{2}$ $\geq$ 0, $y_{3}$ $\geq$ 0, $y_{4}$ $\geq$ 0, $y_{5}$ $\geq$ 0\\

    \end{itemize}

    \underline{Solucion del Modelo}: 

    \begin{itemize}

        \item $y_{1}$ = 400
        \item $y_{2}$ = 0
        \item $y_{3}$ = 1.900
        \item $y_{4}$ = 900
        \item $y_{5}$ = 0
        \item $Z$ = \$19.800.000\\

    \end{itemize}

    \underline{Verificamos si cumple las restricciones}:

    \begin{itemize}

        \item 10 x 400 + 20 x 0 + 1 x 1.900 = 5.900 $\geq$ 5.900
        \item 5 x 400 + 10 x 0 + 1 x 900 = 2.900 $\geq$ 2.900
        \item 3 x 400 + 6 x 0 + 1 x 0 = 1.200 $\geq$ 1.200
        \item Condiciones de no negatividad: 400 $\geq$ 0, 0 $\geq$ 0, 1.900 $\geq$ 0, 900 $\geq$ 0, 0 $\geq$ 0\\

    \end{itemize}

    Entonces es solución factible dual.\\

    \clearpage

    Ahora, para hallar la base del diccionario óptimo del problema sabesmo que:

    \begin{itemize}

        \item Si una \textbf{variable primal} $x_{j}$ $>$ 0 $\rightarrow$ \textbf{esta} en la base.
        \item Si una \textbf{variable primal} $x_{j}$ $=$ 0 $\rightarrow$ \textbf{no esta} en la base.
        \item Si una \textbf{variable dual} $y_{i}$ $>$ 0 $\rightarrow$  entonces la \textbf{restricción primal} 
        i está \textbf{activa}, por lo que la \textbf{variable de holgura asociada} está \textbf{fuera de la base}.
        \item Si una \textbf{variable dual} $y_{i}$ $=$ 0 $\rightarrow$  la restricción primal \textbf{no está activa}, por lo tanto su \textbf{variable de 
        holgura  esta} en la base. \\

    \end{itemize}

    Por lo que las variables quedan de la sigueinte forma:

    \begin{itemize}

        \item Variables Basicas:
        \begin{itemize}

            \item $x_{1}$
            \item $x_{2}$
            \item $x_{3}$
            \item $x_{5}$
            \item $x_{8}$\\
    
        \end{itemize}
        \item Variables No Basicas:
        \begin{itemize}

            \item $x_{4}$
            \item $x_{6}$
            \item $x_{7}$
    
        \end{itemize}

    \end{itemize}

    Entonces tenemos los sigueintes datos:\\

    \begin{center}
    $X_{B} = \{x_{1}, x_{2}, x_{3}, x_{5}, x_{8}\}$ \hspace{1em} $X_{N} = \{x_{4}, x_{6}, x_{7}\}$ 
    \hspace{1em} $C_{B}$ = \{5.900, 2.900, 1.200, 0, 0\} \hspace{1em} $C_{N} = \{0, 0, 0\}$\\
    \end{center}

    \[
    B = \begin{bmatrix}
    10 & 5  & 3 & 0 & 0 \\
    20 & 10 & 6 & 1 & 0 \\
    1  & 0  & 0 & 0 & 0 \\
    0  & 1  & 0 & 0 & 0 \\
    0  & 0  & 1 & 0 & 1 \\
    \end{bmatrix}
    \hspace{2em}
    B^{-1} = \begin{bmatrix}
        0           & 0 &    1	        &   0            &	0 \\
        0           & 0 &    0	        &   1            &	0 \\
      \frac{1}{3}   & 0 & -\frac{10}{3} & -\frac{5}{3}   &	0 \\
       -2           & 1 &    0	        &   0            &	0 \\
       -\frac{1}{3} & 0 & \frac{10}{3}	& \frac{5}{3}    &	1 \\
    \end{bmatrix}
    \]
    \vspace{1em}
    \[
        A = \begin{bmatrix}
        1 & 0 & 0 \\
        0 & 0 & 0 \\
        0 & 1 & 0 \\
        0 & 0 & 1 \\
        0 & 0 & 0 \\
        \end{bmatrix}
        \hspace{2em}
        b = \begin{bmatrix}
            $38.000$ \\
            $80.000$ \\
            $1.000$  \\
            $3.000$  \\
            $7.000$  \\
        \end{bmatrix}
    \]

    \vspace{1em}

    Luego, como se agregan 1.000 horas de extras de refinado, b1 = 39.000. \\
    Ahora hacemos:

    \[
        B^{-1} \times \begin{bmatrix}
            $39.000$ \\
            $80.000$ \\
            $1.000$  \\
            $3.000$  \\
            $7.000$  \\
        \end{bmatrix} = 
        \begin{bmatrix}
            $1.000$ \\
            $3.000$ \\
            $$\frac{14.000}{3}$$  \\
            $2.000$  \\
            $$\frac{7.000}{3}$$  \\
        \end{bmatrix}
    \]

    Da todo positivo, lo que da a entender que la base sigue siendo optima, sin embargo, al correr el modelo en simplex cambindo la variable relacionada al refinado,
    la nueva solucion es:

    \begin{itemize}

        \item $x_{1}$ = 1.000
        \item $x_{2}$ = 3.000
        \item $x_{3}$ = 4.666,666666666667
        \item $Z$ = \$20.200.000\\

    \end{itemize}

    Lo que demuestra que si conviene aceptar esta propuesta.

    Luego, observando que $y_{1}$ = 400, eso quiere decir que la propuesta va a ser rentable siempre que se pague, hasta como mucho, \$400 por cada hora extra del sector. \\

    
    \item Si el director decide pagar por hora extra la mitad del valor máximo indicado en el punto anterior, ¿en cuánto
    aumentaráa la ganacia mensual de la compañía? \\

    Como se puede observar en inciso anterior, el valor dual $y_{1}$ = 400 indica que cada hora adicional de refinado incrementa la ganancia en \$400 (Si
    se obtiene gratis). \\

    Ahora bien, en este caso, el director decide pagar solo la mitad de ese valor dual, es decir, \$200.\\
    Por lo tanto, el \textbf{benefico neto} por cada hora extra sera: \$400 - \$200 = \$200. \\
    Luego, como se compran 1.000 horas extras, el incremento de las ganancia sera: \$200.000.\\

    Es decir, la ganancia mensual de la compañía aumentará en \$200.000 al pagar 1.000 horas extras de refinado a \$200 cada una.

    \item Por otro lado, el gerente de compras propone cambiar algunos proveedores, lo que permitiría bajar el costo
    de la materia prima del aceite para vehículos de \$1000 a \$800 por cada 1000 litros procesados. ¿Cambiaría el
    plan de producción óptimo? Si es así, dar la nueva planificación óptima.

    \item Y si se modificara el proceso de refinado de kerosene para bajar de \$1500 a \$900 por cada 1000 litros procesado,
    ¿cambiaría el plan óptimo? Si es así, dar la nueva planificación óptima.
    
    \item La empresa está evaluando comenzar a procesar gasoil. El tiempo requerido para refinar 1000 litros de gasoil
    es de 4 horas, mientras que para fraccionarlos son necesarias 8 horas y para su embalaje 1.5 horas. El costo
    de la materia prima para mil litros de gasoil es de \$4000, el de refinado de \$4100, el de fraccionado de \$1000.
    El embalaje de gasoil lo realizaría el sector de embalaje de kerosene. ¿Cuál debería ser el menor precio de
    venta de los 1000 litros de gasoil para que su producción sea conveniente para la empresa?
    
    \item La empresa va a agregar un control de calidad a todos sus productos. Controlar los 1000 litros de combustible
    para aviones requiere 5 horas, los de combustible para vehículos 3 horas y 2 horas los 1000 litros de kerosene.
    Si el sector de control de calidad dispone de 20000 horas mensuales, ¿cambiaría el plan óptimo? Si es así, dar
    la nueva planificación óptima.

\end{enumerate}

\clearpage


\end{document}